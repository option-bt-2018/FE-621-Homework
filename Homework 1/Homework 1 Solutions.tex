\documentclass[10pt]{article}

%==============================
% Document Metadata
%============================== 
\usepackage[pdftex,
    pdfauthor={Rukmal Weerawarana},
    pdftitle={Homework 1 Solutions - FE 621},
    pdfsubject={FE 621 - Computational Methods in Finance}
]{hyperref}


%==============================
% Package Imports
%==============================    
\usepackage[ruled]{algorithm2e}
\usepackage[authordate]{biblatex-chicago}
\usepackage{amsmath} % math environment stuff
\usepackage{amssymb} % additional math symbols
\usepackage[toc, page]{appendix} % Appendix referencing
\usepackage{comment} % enables the use of multi-line comments (\ifx \fi) 
\usepackage{csvsimple} % CSV to Table
\usepackage{fancyhdr} % Header
\usepackage[headings]{fullpage}
\usepackage{listings} % code embedding
\usepackage[dvipsnames]{xcolor} % colors for code


%==============================
% Configuration
%==============================

% Bibliography configuration
\addbibresource{../bibliography.bib}

% Code display configuration
\newcommand*\lstinputpath[1]{\lstset{inputpath=#1}} % Setting path
\lstset{
	language=Python,
	basicstyle=\footnotesize\ttfamily,
	commentstyle=\ttfamily\color{purple!40!black},
	identifierstyle=\color{blue},
	keywordstyle=\color{ForestGreen},
	numbers=left,
	numberstyle=\ttfamily\color{gray}\footnotesize,
	stepnumber=1,
	numbersep=5pt,
	backgroundcolor=\color{white},
	showspaces=false,
	showstringspaces=false,
	showtabs=false,
	frame=single,
	tabsize=2,
	captionpos=b,
	breaklines=true,
	breakatwhitespace=false,
	title=\lstname
}
\lstset{
	language=R,
	basicstyle=\footnotesize\ttfamily,
	commentstyle=\ttfamily\color{purple!40!black},
	identifierstyle=\color{blue},
	keywordstyle=\color{ForestGreen},
	numbers=left,
	numberstyle=\ttfamily\color{gray}\footnotesize,
	stepnumber=1,
	numbersep=5pt,
	backgroundcolor=\color{white},
	showspaces=false,
	showstringspaces=false,
	showtabs=false,
	frame=single,
	tabsize=2,
	captionpos=b,
	breaklines=true,
	breakatwhitespace=false,
	title=\lstname
}

% Header and Footer configuration
\pagestyle{fancy} % set page style
\fancyhead{} % override header
\fancyfoot{} % override footer
\renewcommand{\headrulewidth}{.4pt} % set header rule width 
\renewcommand{\footrulewidth}{.4pt} % set footer rule width 
\lhead{Homework Assignment 1} % set left header
\rhead{Rukmal Weerawarana} % set right header
\lfoot{\textit{FE 621}: Computational Methods in Finance} % set left footer
\rfoot{Page \thepage} % set right footer


%==============================
% Document Content
%==============================

\begin{document}

\thispagestyle{plain}

%==============================
% Document Title
%==============================

\noindent
\large\textbf{Homework Assignment 1} \hfill \textbf{Rukmal Weerawarana} \\
\normalsize \textit{FE 621}: Computational Methods in Finance \hfill \textit{rweerawa@stevens.edu} $\mid$ 104-307-27 \\
\textit{Instructor}: Ionut Florescu \hfill Department of Financial Engineering \\
2/20/2019 \hfill Stevens Institute of Technology

\noindent\rule{\linewidth}{.1em}


%==============================
% Overview
%==============================

\section*{Overview}

In this Homework Assignment, we explore various numerical optimization methods through the lens of the Black-Scholes-Merton Option pricing model (\cite{Shreve2004}). Using this, we calculate explore the implied volatility of options for various assets traded on the market. Furthermore, we also explore numeric methods of differential calculation to compute the Greeks of these candidate options. Finally, we explore numeric integration and the behavior of various quadrature methods.

Unless otherwise stated, the following shorthand notation is used to distinguish between dates:

\begin{itemize}
    \item \textbf{DATA1} - Wednesday, February 6 2019 (\textit{2/6/19})
    \item \textbf{DATA2} - Thursday, February 7 2019 (\textit{2/7/19})
\end{itemize}

The content of this Homework Assignment is divided into three sections; the first discusses data gathering, formatting, and a discussion of the assets being examined. The second contains data analysis, and an exploration of implied volatility through the Black-Scholes-Merton pricing framework and related computations. Finally, the third section discusses numerical integration and the convergence of various quadrature rules.

\begin{center}
    \textit{See Appendix \ref{appendix:source} for specific question implementations, and (\cite{Weerawarana2019}) for full source code.}
\end{center}


%==============================
% Section 1
%==============================

\newpage

\section{Data Overview}

    \subsection{Asset Descriptions}

        \subsubsection{\textit{SPY} - SPDR S\&P 500 ETF (\cite{StateStreetGlobalAdvisors2019})}

        The S\&P 500 (i.e. \textit{Standard \& Poor's 500}) is a stock market index tracking the 500 largest companies on the American Stock Exchange by Market Capitalization. In this case, the market capitalization is defined as the number of outstanding shares, multiplied by the current share price. A stock market index is designed to be a metric that can be used by market observers as a benchmark to gauge the relative health of the stock market, by analyzing the aggregate performance of its largest components.
            
        However, this index is not the same as the \textit{SPY} ETF. An ETF (\textit{Exchange Traded Fund}) is a basket of stocks that is designed to track a specific index or benchmark. That is, it provides investors with exposure to a index or benchmark, without having to own all of the underlying assets that constitute a composite asset. In addition to more liquidity, this type of investment also provides lower transaction costs and required minimum investment to gain exposure to a given index or benchmark. It is traded on an exchange, akin to a typical traded asset.

        \subsubsection{\textit{VIX} - CBOE Volatility Index (\cite{CBOEChicagoBoardOptionsExchange2019})}

        The CBOE (\textit{Chicago Board Options Exchange}) volatility index, \textit{VIX} is an exchange traded product (\textit{ETP}) designed to give investors exposure to the market's expectation of 30-day volatility. It is priced using a large set of implied volatility of put and call options on the S\&P 500 index to gauge investor sentiment. Typically, the price of the VIX has an inverse relationship to the price of the S\&P 500 index. Similar to an ETF, an ETP is also traded on an exchange as a typical traded asset.


%==============================
% References
%==============================

\newpage

\printbibliography


%==============================
% Appendix
%==============================

\newpage

\appendix

% Resetting input path
\lstinputpath{question_solutions}

\section{Appendix} \label{appendix:source}

    \subsection{Question 1 Implementation}
    \label{appendix:source:q1}

        \lstinputlisting{question_1.R}

%==============================
% Document End
%==============================

\end{document}
