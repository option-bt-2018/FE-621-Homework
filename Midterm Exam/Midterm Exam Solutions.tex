\documentclass[10pt]{article}

%==============================
% Document Metadata
%============================== 
\usepackage[pdftex,
    pdfauthor={Rukmal Weerawarana},
    pdftitle={Homework 1 Solutions - FE 621},
    pdfsubject={FE 621 - Computational Methods in Finance}
]{hyperref}


%==============================
% Package Imports
%==============================    
\usepackage[ruled]{algorithm2e} % typeset algorithms
\usepackage[authordate, maxcitenames=1]{biblatex-chicago} % chicago bibliography style
\usepackage{amsmath} % math environment stuff
\usepackage{amssymb} % additional math symbols
\usepackage[toc, page]{appendix} % Appendix referencing
\usepackage{booktabs} % Table lines
\usepackage{comment} % enables the use of multi-line comments (\ifx \fi)
\usepackage[skip=5pt, labelfont=bf]{caption} % caption formatting
\usepackage{csvsimple} % CSV import to Table
\usepackage{enumitem} % Lists with alphabetical bullet points
\usepackage{fancyhdr} % Header
\usepackage{fancyvrb} % Verbatim text
\usepackage{float} % Controlling figure border
\usepackage[headings]{fullpage} % Set all margins to 1.5 cm
\usepackage{graphicx} % Figures
\usepackage{listings} % code embedding
\usepackage{longtable} % Multipage tables
\usepackage{multirow} % Multirow cells in tables
\usepackage{pmboxdraw} % Box characters for file tree
\usepackage[dvipsnames]{xcolor} % colors for code


%==============================
% Configuration
%==============================

% Figure outline configuration
% \floatstyle{boxed}
% \restylefloat{figure}

% Bibliography configuration

\addbibresource{../bibliography.bib}

% Remapping bibliography underscores (_) and tildes (~) because Mendeley has weird exporting
% Solution from: https://tex.stackexchange.com/questions/309980/parsing-underscores-in-urls-from-mendeley

\DeclareSourcemap{ % Used when .bib/Bibliography is compiled, not when document is
    \maps{
        \map{ % Replaces '{\_}', '{_}' or '\_' with just '_'
            \step[fieldsource=url,
                  match=\regexp{\{\\\_\}|\{\_\}|\\\_},
                  replace=\regexp{\_}]
        }
        \map{ % Replaces '{'$\sim$'}', '$\sim$' or '{~}' with just '~'
            \step[fieldsource=url,
                  match=\regexp{\{\$\\sim\$\}|\{\~\}|\$\\sim\$},
                  replace=\regexp{\~}]
        }
    }
}

% Code display configuration

\newcommand*\lstinputpath[1]{\lstset{inputpath=#1}} % Setting path
\lstset{
	language=Python,
	basicstyle=\footnotesize\ttfamily,
	commentstyle=\ttfamily\color{purple!40!black},
	identifierstyle=\color{blue},
	keywordstyle=\color{ForestGreen},
	numbers=left,
	numberstyle=\ttfamily\color{gray}\footnotesize,
	stepnumber=1,
	numbersep=5pt,
	backgroundcolor=\color{white},
	showspaces=false,
	showstringspaces=false,
	showtabs=false,
	frame=single,
	tabsize=2,
	captionpos=b,
	breaklines=true,
	breakatwhitespace=false,
	title=\lstname
}
\lstset{
	language=R,
	basicstyle=\footnotesize\ttfamily,
	commentstyle=\ttfamily\color{purple!40!black},
	identifierstyle=\color{blue},
	keywordstyle=\color{ForestGreen},
	numbers=left,
	numberstyle=\ttfamily\color{gray}\footnotesize,
	stepnumber=1,
	numbersep=5pt,
	backgroundcolor=\color{white},
	showspaces=false,
	showstringspaces=false,
	showtabs=false,
	frame=single,
	tabsize=2,
	captionpos=b,
	breaklines=true,
	breakatwhitespace=false,
	title=\lstname
}

% Header and Footer configuration

\pagestyle{fancy} % set page style
\fancyhead{} % override header
\fancyfoot{} % override footer
\renewcommand{\headrulewidth}{.4pt} % set header rule width 
\renewcommand{\footrulewidth}{.4pt} % set footer rule width 
\lhead{Midterm Examination} % set left header
\rhead{Rukmal Weerawarana} % set right header
\lfoot{\textit{FE 621}: Computational Methods in Finance} % set left footer
\rfoot{Page \thepage} % set right footer


%==============================
% Document Content
%==============================

\begin{document}

\thispagestyle{plain}

\pagenumbering{roman}  % Changing numbering to Roman numerals for first pages

%==============================
% Document Title
%==============================

\noindent
\large\textbf{Midterm Examination} \hfill \textbf{Rukmal Weerawarana} \\
\normalsize \textit{FE 621}: Computational Methods in Finance \hfill \textit{rweerawa@stevens.edu} $\mid$ 104-307-27 \\
\textit{Instructor}: Ionut Florescu \hfill Department of Financial Engineering \\
3/31/2019 \hfill Stevens Institute of Technology

\noindent\rule{\linewidth}{.1em}


%==============================
% Overview
%==============================

\section*{Overview}

This is my solution manuscript for the FE 621 Midterm Examination.

Unless otherwise stated, all solutions in the manuscript assume that the number of days in a year to be 365.

The content of this manuscript is divided into four sections; the first addresses Problem 1 (Numerical Integration). The second section addresses Problem 2, pricing options with Trinomial Additive Trees. The third section addresses Problem 3, and computes various ranges of possible option prices. Finally, the fourth section answers Problem 4, and discretizes a partial differential equation describing a stochastic option pricing model.

\begin{center}
    \textit{See Appendix~\ref{appendix:source} for specific question implementations, and the project GitHub repository\footnote{\cite{Weerawarana2019}} for full source code of the {\normalfont \texttt{fe621}} Python package.}
\end{center}


%==============================
% Table of Contents
%==============================

\newpage

\tableofcontents

%==============================
% SOLUTIONS
%==============================

\newpage

\pagenumbering{arabic}  % Changing numbering to arabic numerals for main content

%==============================
% Question 1
%==============================

\section{Question 1}

\fbox{
    \begin{minipage}{\linewidth}
        \textbf{Problem 1: Numerical Integration}

        \begin{enumerate}[label=(\alph*)]
            \item Numerically compute the integral $\int_0^2 e^{x^2} \, dx$ using the trapezoid method with 100 steps.
            \item Numerically compute the integral $\int_0^2 e^{x^2} \, dx$ using Simpson's quadrature rule, with 100 steps.
            \item Please compare the two results obtained. Comment.
        \end{enumerate}
    \end{minipage}
}

\subsection{Part (a) and (b)}

To answer this question, I am using implementations of Simpson's and Trapezoidal Quadrature rules from the \texttt{fe621} package. The complete methodology for these two quadrature rules are outlined in Homework 1.\footnote{\cite{Weerawarana2019}} For clarity, the quadrature rule approximation equations for both Simpson's rule and the Trapezoidal rule, $S_N(f)$ and $T_N(f)$, respectively, are reproduced below:

\begin{gather*}
    \text{Let data} = \boldsymbol{x} \\
    \text{Let $i^\text{th}$ element of $\boldsymbol{x}$} = x_i \\
    \boldsymbol{x}_\text{mid} = \left( \frac{x_{i - 1} + x_i}{2} \right) \\
    \\
    \Rightarrow S_N(f) \approx \frac{h}{6} (2f(\boldsymbol{x}) - (f(x_0) + f(x_N)) + 4f(\boldsymbol{x}_\text{mid})) \\
    \Rightarrow T_N(f) = hf(\boldsymbol{x}) - \frac{h}{2} (f(x_0) + f(x_N))
\end{gather*}

To compute the integral in the question, implementations from the \texttt{fe621} package was used for both of the quadrature rules. The source code for both Simpson's and Trapezoidal quadrature rules are reproduced in Appendix~\ref{appendix:fe621:simpsons} and Appendix~\ref{appendix:fe621:trapezoidal}, respectively. Furthermore, the source code for the computation and table output of the solutions to \textit{Part (a)} and \textit{Part (b)} is reproduced in Appendix~\ref{appendix:source:q1}.

\newpage
\subsection{Part (c)}

\begin{table}[!h]
    \centering
    \csvautotabular{bin/question_1.csv}
    \caption{Numerical approximations for the integral $\int_0^2 e^{x^2} \, dx$ with the Simpson's and Trapezoidal quadrature rules.}
    \label{table:quadrature_rule_approximations}
\end{table}

The approximations for the integral $\int_0^2 e^{x^2} \, dx$ under both Simpson's and Trapezoidal quadrature rules are reproduced in Table~\ref{table:quadrature_rule_approximations}.

It can be seen that the approximated integrals under both of the quadrature rules are extremely close, differing by $< 0.01$. This indicates convergence under the two rules, and may be attributed to the large number of steps, $N = 100$, for the small interval 0 to 2.

Utilizing the Wolfram|Alpha computation platform\footnote{\cite{Wolfram|Alpha2019}}, I found that the true estimated value of the integral $\approx 16.4526$. Analyzing the computed approximations through the lens of this solution, it is clear that the Simpson's Quadrature Rule assumption is significantly closer to the Wolfram|Alpha approximation, compared to the Trapezoidal Quadrature Rule.

A potential explanation of this may be the interpolating behavior of Simpson's Quadrature Rule. The function $e^{x^2}$ is better approximated quadratically than linearly in small intervals between 0 and 2. Thus, the quadratic interpolating behavior of Simpson's Rule is a better approximation heuristic for the function with 100 steps. Despite this shortcoming, the Trapezoidal quadrature rule approximation is also extremely close to the Wolfram|Alpha computed area. Theoretically, both approximations will converge with the true value as the number of steps, $N \rightarrow \infty$.



%==============================
% Question 2
%==============================
\newpage
\section{Question 2}

\fbox{
    \begin{minipage}{\linewidth}
        \textbf{Problem 2: Option Pricing using a Trinomial Tree}

        Construct a Trinomial tree to price an American put option. To this end, start with the following given parameters: $S_0 = 100$, $K = 120$, maturity $T = 8$ months, $r = 0$, $\delta = 0$, volatility $\sigma = 30\%$, and time steps $N = 200$.

        \begin{enumerate}[label=(\alph*)]
            \item What is the best choice for $\Delta x$ to obtain the best order of convergence? Calculate $\Delta x$.
            \item Calculate the American Put option price using the tree.
            \item Estimate Gamma of the American Put at time $t = 0$.
        \end{enumerate}
    \end{minipage}
}

\vspace{2em}

For this question, I will be using the \texttt{AdditiveTree} Trinomial tree model, outlined in Homework 2.\footnote{\cite{Weerawarana2019}} This builds on the \texttt{GeneralTree} generalized tree implementation, also discussed in Homework 2. The source code for these two modules from the \texttt{fe621} package are reproduced in Appendix~\ref{appendix:fe621:additivetree} and Appendix~\ref{appendix:fe621:generaltree}, respectively.


\subsection{Part (a)}

In order to guarantee a convergent trinomial tree, the following condition was imposed on the jump of each step on the tree, $\Delta x$\footnote{\cite{Florescu2019a}}:

\begin{gather*}
    \Delta x \geq \sigma \sqrt{3 \Delta t}
\end{gather*}

Utilizing the option parameters specified in the question, the lower bound on the jump, $\Delta x$ can be computed:

\begin{gather*}
    \Delta t = T / N = \frac{8}{12} \cdot \frac{1}{200} \\
    \sigma = 0.3 \\
    \Rightarrow \Delta x
        = \sigma \sqrt{3 \Delta t}
        = 0.3 \sqrt{3 \cdot \frac{8}{12} \cdot \frac{1}{200}}
        = 0.3 \sqrt{\frac{3}{200} \cdot \frac{2}{3}}
        = 0.3 \sqrt{0.01} \\
    \therefore \Delta x = 0.03
\end{gather*}


\subsection{Part (b)}

Utilizing the jump size computed above, an Additive Trinomial Tree was used to compute the price of an American Put option with the parameters stated in the question:

\begin{table}[!h]
    \centering
    \csvautotabular{bin/question_2_price.csv}
    \caption{Price of an American Put option, computed with a Trinomial Additive Tree.}
    \label{table:q2_american_put_price}
\end{table}

\newpage
\subsection{Part (c)}

To compute the Gamma, $\Gamma$, of the American Put option at time $t = 0$, the Central Finite Difference Method was used. $\Gamma$ is defined as the second derivative of the option value, $V$, with respect to the underlying asset price, $S$:

\begin{gather*}
    \Gamma = \frac{\partial^2 V}{\partial S^2}
\end{gather*}

The second-order central finite difference method for an arbitrary three-times differentiable function, $f$, in an interval around the point $a$ is\footnote{\cite{Stefanica2011}}:

\begin{gather*}
    \text{Let} \, h > 0 \\
    \Rightarrow f^{\prime\prime} (a) \approx \frac{f(a + h) - 2f(a) + f(a - h)}{h^2} + O(h^2)
\end{gather*}


This can be applied to the tree-pricing methodology utilized to compute the initial price for the American Put option. By setting $f(a)$ to be equal to the trinomial-tree computed price of the option, given initial stock price $a$, an approximation for the Gamma, $\Gamma$ of the option can be computed.

To accomplish this, the central finite difference method from the \texttt{fe621} package (outlined in Homework 1) was utilized. The source code for this approximation method is reproduced in Appendix~\ref{appendix:fe621:2cfd}. As with the previous question, the source code for this computation is reproduced in Appendix~\ref{appendix:source:q2}.

\begin{table}[!h]
    \centering
    \csvautotabular{bin/question_2_gamma.csv}
    \caption{Estimated Gamma, $\Gamma$ of the American Put at time $t = 0$.}
    \label{table:q2_american_put_gamma}
\end{table}


%==============================
% Question 3
%==============================
\newpage
\section{Question 3}

\fbox{
    \begin{minipage}{\linewidth}
        \textbf{Problem 3: Option Price Range}

        Assume a stock is at \$23.35. You look at the market to a European Call option with strike \$22.50 and maturity of 8 weeks. Assume $r = 0.01$. The listed best bid is \$3.20, and the best ask is \$3.80. Use the code you turned in the assignments to answer the following questions:

        \begin{enumerate}[label=(\alph*)]
            \item Calculate an interval of possible values for the European Put.
            \item Calculate an interval of possible values for the American Call.
        \end{enumerate}
    \end{minipage}
}

\vspace{2em}

\textbf{Note:} For this question, I am assuming that the instructor wants us to compute an interval of possible option prices, utilizing the best bid and ask values as upper and lower bounds on the price of the original call option. I am making this assumption due to the fact that the instructor has not provided corresponding volumes for the best bid and ask offer values, thus making a volume-weighted average price computation impossible.

Under this assumption, I computed upper and lower bounds on the implied volatility of the option, using the best bid and ask price of the call option, in conjunction with a Bisection-method optimization on the Black-Scholes option pricing formula. The corresponding \texttt{fe621} package source code for the Black-Scholes Call Price and the Bisection Optimization Method are reproduced in Appendix~\ref{appendix:fe621:bs_call}, and Appendix~\ref{appendix:fe621:bisection}, respectively.

The source code for this computation is reproduced in Appendix~\ref{appendix:source:q3}. Utilizing the assumption outlined above, upper and lower bounds on the implied volatility were computed and are reproduced in Table~\ref{table:q3_imp_vol_bounds}.

\begin{table}[!h]
    \centering
    \csvautotabular{bin/question_3_imp_vol.csv}
    \caption{Upper and lower bounds on the implied volatility of the European Call Option, using the Best Bid as the Lower Bound price, and Best Ask as the Upper Bound price for the Bisection optimizer.}
    \label{table:q3_imp_vol_bounds}
\end{table}


\newpage
\subsection{Part (a)}

The upper and lower bounds on the implied volatility (see Table~\ref{table:q3_imp_vol_bounds}) were used to compute a range of possible option prices for a European Put option, using the Black-Scholes formula.

The \texttt{fe621} package source code for computing the Black-Scholes European Put option price is reproduced in Appendix~\ref{appendix:fe621:bs_put}. The source code for this computation is reproduced in Appendix~\ref{appendix:source:q3}. The upper and lower bounds on the price of a European Put option are reproduced in Table~\ref{table:q3_eu_put_prices}.

\begin{table}[!h]
    \centering
    \csvautotabular{bin/question_3_eu_put_prices.csv}
    \caption{Upper and lower bounds on the price of a European Put option.}
    \label{table:q3_eu_put_prices}
\end{table}


\subsection{Part (b)}

Similar to Part (a), the upper and lower bounds on the implied volatility (see Table~\ref{table:q3_imp_vol_bounds}) were used to compute a range of possible prices for an American Call option, using the Trigeorgis Binomial Pricing Tree.

The \texttt{fe621} package source code for the Trigeorgis tree, and the \texttt{GeneralTree} generalized tree on which it is based is reproduced in Appendix~\ref{appendix:fe621:trigeorgis}, and Appendix~\ref{appendix:fe621:generaltree}, respectively. The source code for this computation is reproduced in Appendix~\ref{appendix:source:q3}. The upper and lower bounds on the price of an American Call option are reproduced in Table~\ref{table:q3_american_call_prices}.

\begin{table}[!h]
    \centering
    \csvautotabular{bin/question_3_american_call_prices.csv}
    \caption{Upper and lower bounds on the price of an American Call option.}
    \label{table:q3_american_call_prices}
\end{table}


%==============================
% Question 4
%==============================
\newpage
\section{Question 4}

\fbox{
    \begin{minipage}{\linewidth}
        \textbf{Problem 4}

        We know that an option price under a certain stochastic model satisfies the following PDE:

        \begin{center}
            $\frac{\partial V}{\partial t} + 2\cos{(S)} \frac{\partial V}{\partial S} + 0.2 S^{\frac{3}{2}} \frac{\partial^2 V}{\partial S^2} - rV = 0$.
        \end{center}

        Assume you have an equidistant grid with points of the form $(i, j) = (i \Delta t, j \Delta x)$, where $i \in \{1, 2, \ldots, N\}$, and $j \in \{-N_S, N_S\}$. Let $V_{i,j} = (i \Delta t, j \Delta x)$. Discretize the derivatives and give finite difference equation for an Explicit scheme. Use the notation introduced above.
    \end{minipage}
}


\begin{gather*}
    \frac{\partial V}{\partial t} + 2\cos{(S)} \frac{\partial V}{\partial S} + 0.2 S^\frac{3}{2} \frac{\partial^2 V}{\partial S^2} - rV = 0 \\
    \\
    \text{Let } \ln{(S) = x} \\
    \Rightarrow S = e^x \\
    \Rightarrow \frac{dS}{dx} = e^x \\
    \therefore dS = e^x dx = S dx \\
    \\
    \text{We can use this substitution to reorganize the initial PDE:} \\
    - \frac{\partial V}{\partial t} = \frac{2\cos{(S)}}{S} \frac{\partial V}{\partial x} + \frac{0.2}{\sqrt{S}} \frac{\partial^2 V}{\partial x^2} - rV \\
    \\
    \text{We can discretize the stock process, $S$, and the value of the option, $V$.} \\
    \\
    \text{Utilizing finite difference methods:} \\
    \Rightarrow \frac{\partial V_{i, j}}{\partial t} = \frac{V_{i+1, j} - V_{i, j}}{\Delta t} \\
    \Rightarrow \frac{\partial V_{i, j}}{\partial x} = \frac{V_{i+1,j+1} - V_{i+1, j-1}}{2 \Delta x} \\
    \Rightarrow \frac{\partial^2 V_{i, j}}{\partial x^2} = \frac{V_{i+1, j+1} - 2V_{i+1, j} + V_{i+1, j-1}}{\Delta x^2} \\
    \\
    \text{Substituting this in the reorganized discretized PDE:} \\
    - \frac{\partial V}{\partial t} = \frac{2\cos{(S)}}{S} \frac{\partial V}{\partial x} + \frac{0.2}{\sqrt{S}} \frac{\partial^2 V}{\partial x^2} - rV \\
    \\
    \Rightarrow - \frac{V_{i+1, j} - V_{i, j}}{\Delta t} = \frac{2\cos{(S_{i, j})}}{S_{i,j}} \frac{V_{i+1,j+1} - V_{i+1, j-1}}{2 \Delta x} + \frac{0.2}{\sqrt{S_{i, j}}} \frac{V_{i+1, j+1} - 2V_{i+1, j} + V_{i+1, j-1}}{\Delta x^2} - rV_{i+1, j}
\end{gather*}

\newpage
\begin{gather*}
    \text{We can now reorganize the equation, and solve for $V_{i, j}$} \\
    - \frac{V_{i+1, j} - V_{i, j}}{\Delta t} = \frac{2\cos{(S_{i, j})}}{S_{i,j}} \frac{V_{i+1,j+1} - V_{i+1, j-1}}{2 \Delta x} + \frac{0.2}{\sqrt{S_{i, j}}} \frac{V_{i+1, j+1} - 2V_{i+1, j} + V_{i+1, j-1}}{\Delta x^2} - rV_{i+1, j} \\
    \Rightarrow V_{i, j} = \Delta t \left( \frac{2\cos{(S_{i, j})}}{S_{i,j}} \frac{V_{i+1,j+1} - V_{i+1, j-1}}{2 \Delta x} + \frac{0.2}{\sqrt{S_{i, j}}} \frac{V_{i+1, j+1} - 2V_{i+1, j} + V_{i+1, j-1}}{\Delta x^2} - rV_{i+1, j} \right) + V_{i+1, j} \\
    \\
    \text{Reorganizing the equation to isolate $p_u$, $p_m$, and $p_d$:} \\
    \Rightarrow V_{i, j} = V_{i+1,j+1}\left( \frac{\Delta t 0.2}{\sqrt{S_{i,j}} \Delta x^2} + \frac{2 \Delta t \cos{(S_{i,j})}}{2\Delta x S_{i,j}} \right) + V_{i+1,j} \left( 1 - r \Delta t - \frac{2\Delta t 0.2}{\sqrt{S_{i,j}} \Delta x^2} \right) \\
    + V_{i+1,j-1} \left( \frac{\Delta t 0.2}{\sqrt{S_{i,j}} \Delta x^2} - \frac{2\Delta t \cos{(S_{i,j})}}{2\Delta x S_{i,j}} \right) \\
    \\
    \text{From the equation above, we can isolate the jump probabilities of the tree:} \\
    p_u
        = \frac{\Delta t 0.2}{\sqrt{S_{i,j}} \Delta x^2} + \frac{2 \Delta t \cos{(S_{i,j})}}{2\Delta x S_{i,j}}
        = \frac{\Delta t}{\sqrt{S_{i,j}} \Delta x} \left( \frac{0.2}{\Delta x} + \frac{\cos{(S_{i,j})}}{\sqrt{S_{i,j}}} \right) \\
    p_m
        = 1 - r \Delta t - \frac{2\Delta t 0.2}{\sqrt{S_{i,j}} \Delta x^2}
        = 1 - \Delta t \left( r + \frac{0.4}{\sqrt{S_{i,j}} \Delta x^2} \right) \\
    p_d
        = \frac{\Delta t 0.2}{\sqrt{S_{i,j}} \Delta x^2} - \frac{2\Delta t \cos{(S_{i,j})}}{2\Delta x S_{i,j}}
        = \frac{\Delta t}{\sqrt{S_{i,j}} \Delta x} \left( \frac{0.2}{\Delta x} - \frac{\cos{(S_{i,j})}}{\sqrt{S_{i,j}}} \right) \\
    \\
    \therefore V_{i,j} = p_u \cdot V_{i+1,j+1} + p_m \cdot V_{i+1,j} + p_d \cdot V_{i+1,j-1}
\end{gather*}

%==============================
% REFERENCES
%==============================

\newpage

\printbibliography

%==============================
% APPENDIX
%==============================

\newpage

\appendix

% Resetting input path
\lstinputpath{}

\newpage
\section{Solution Source Code} \label{appendix:source}
    \subsection{Question 1 Solution} \label{appendix:source:q1}
        \lstinputlisting{question_solutions/question_1.py}

    \subsection{Question 2 Solution} \label{appendix:source:q2}
        \lstinputlisting{question_solutions/question_2.py}
    
    \subsection{Question 3 Solution} \label{appendix:source:q3}
        \lstinputlisting{question_solutions/question_3.py}

\newpage
\section{\texttt{fe621} Package Code} \label{appendix:fe621}
% Setting path to parent directory
\lstinputpath{..}
    \subsection{Simpson's Quadrature Rule} \label{appendix:fe621:simpsons}
        \lstinputlisting{fe621/numerical_integration/simpsons.py}

    \subsection{Trapezoidal Quadrature Rule} \label{appendix:fe621:trapezoidal}
        \lstinputlisting{fe621/numerical_integration/trapezoidal.py}

    \subsection{Additive Tree Trinomial Tree} \label{appendix:fe621:additivetree}
        \lstinputlisting{fe621/tree_pricing/trinomial/trinomial_price.py}
    
    \subsection{\texttt{GeneralTree} Generalized Tree} \label{appendix:fe621:generaltree}
        \lstinputlisting{fe621/tree_pricing/general_tree.py}
    
    \subsection{Second-Order Central Finite Difference} \label{appendix:fe621:2cfd}
        \lstinputlisting{fe621/numerical_differentiation/second_derivative.py}

    \subsection{Black-Scholes Call Option Price} \label{appendix:fe621:bs_call}
        \lstinputlisting{fe621/black_scholes/call.py}

    \subsection{Bisection Method Optimizer} \label{appendix:fe621:bisection}
        \lstinputlisting{fe621/optimization/bisection.py}
    
    \subsection{Black-Scholes Put Option Price} \label{appendix:fe621:bs_put}
        \lstinputlisting{fe621/black_scholes/put.py}
    
    \subsection{Trigeorgis Binomial Tree} \label{appendix:fe621:trigeorgis}
        \lstinputlisting{fe621/tree_pricing/binomial/trigeorgis.py}

%==============================
% Document End
%==============================

\end{document}
